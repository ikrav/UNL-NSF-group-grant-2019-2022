\section{Broader Impacts of the Proposed Work}

%XXX Anyone: ok to mention underrepresented minorities below even though BEST is gone?
%XXX This paragraph is kept from the last proposal because it is good and does not need changes.
The UNL HEP group leads a number of activities that provide broader impacts beyond the intellectual merit already described. Some of these are unique within the field or have inspired similar efforts elsewhere. The outcomes of these programs include  greater participation of underrepresented minorities in STEM fields, increased public scientific literacy and public engagement with science and technology, and the improved well-being of individuals in society.

\paragraph{CMS Education and Outreach}
UNL staff have made an outstanding contribution to international outreach
efforts at CERN as described in Section~\ref{sec:prior}.
Postdocs and graduate students stationed at CERN will continue this work in the
next several years. Planned activities are:
% XXX This bullet list can be converted to plane text if space is an issue.
% XXX KB -- I'm noticing the above comment while on a plane!
\begin{packed_itemize}

     \item Conduct virtual tours at CERN.  UNL
   	postdocs will continue to serve as guides in the virtual tours offered by CERN
       Outreach. The tours involve international participants, usually high school
       students or the general public, video-conferencing with a guide at CERN.
       The guide visits facilities with a cameraman, and gives
       explanations and answers questions in real time.
       %The virtual tours are open to anyone, and members of the general public from U.S. have already
       %taken advantage of them. %Each guide conducts several virtual tours per year.

     \item Conduct on-site facility tours at CERN for members of the public. Our postdocs will continue to serve as CERN-certified guides for CMS,
       other CERN experiments, and the accelerator facilities. Guides provide of order one facility tour per month (several per month during
       LHC shutdown periods). Many visitors from the U.S. benefit from these tours.

   \item Our postdocs are expected to be participants in short outreach videos
       produced by CMS/CERN and U.S.~CMS/FNAL outreach, as they have been in past years.
\end{packed_itemize}

%XXX Dan, please review commented out text below. Is it still relevant? Anything happened on that? Not relevant anymore?
% At UNL, our group is also working with the university's Sheldon Museum of Art to explore hosting an ``art@CMS"~\cite{bib:artcms} exhibition in the museum a year or more from now.

\noindent
%XXX TODO Greg: please check the paragraph below. It is meant to just highlight
%  the broader impact of this work, the actual description is in the "future" section.
Snow will continue his leadership in the CMS Thesis Award and the CMS Named Lectures programs already described in detail in Section~\ref{sec:prior}. Recognition through such an award or a named lecture enhances motivation of the best CMS graduate students and postdocs and adds to their CVs, helping to build a career they deserve.

The group's faculty will provide UNL undergraduate students with a unique research experiences. Claes and Kravchenko will oversee several physics majors contributing to simple testing equipment software development, electronics circuit design, and construction of Phase~2 TFPX modules in the group's silicon lab.
%Students interested in computing could work with Bloom on a variety of projects in computing monitoring and data management.  
Bloom will offer a variety of projects in computing monitoring and data management to students interested in computing.
The importance of undergraduate research cannot be overemphasized: it improves retention rates, helps students become better future 
scientists and professionals, and tremendously increases their motivation.

%XXX Dan, please revise the CROP paragraph and your add other activities as separate paragraphs 
\paragraph{CROP}
For over 18 years, we have engaged teams of high school teachers and students in hands-on research experience studying extended cosmic ray air showers though the Cosmic Ray Observatory Project (CROP). When funded by separate grants through the NSF Division of Research on Learning in Formal and Informal Settings an expanding group of Nebraska high school teams were trained in the construction and operation of school-based scintillation detectors through summer workshops, academic year conferences, and video-streamed after school meetings. Effective relationships were established with Educational Service Unit (ESU) administrators across the state, critical to broadening our reach. A network of local tech industries (across multiple ESUs) informed our goals and provided speakers who joined students in workshop activities and conducted sessions on careers. Marked increases were found for all groups of participants (male/female, rural/urban, student/teacher) in a number of content assessments. Rural teams proved as interested and proficient as urban (key to the remote training delivery we were striving to develop). By adjusting to meet the needs expressed in focus groups, participating students and teachers returned and new schools joine in each year the program was funded.
Participating teachers were interviewed extensively as part of a May 2016 PhD thesis exploring the impact of CROP on their classrooms, with surprising findings on the development of teacher participant’s physics identity~\cite{bib:teacherdevelopment}.

Through the group’s existing NSF grant and UNL's Undergraduate Creative Activities and Research Experiences Program  Fellowships, the UNL HEP group has provided continuing support for a small team of undergraduate physics majors trained as research workers.  For some, work on detectors and software for CROP has served as a gateway to projects associated with CMS.  But many of them build their experience over years with the project, and become indispensable members of the CROP team.  These undergraduates test production runs of the Quarknet data acquisition circuit boards produced in collaboration with Fermilab, and debug, modify, and rewrite the Labview software that runs it.  Their hardware experience includes scintillator preparation (sawing, sanding, polishing), detector building (PMT mounting), power supply assembly, oscilloscope use, and use of the DAQcard. They pre-test and select PMTs, as well as determine appropriate discriminator threshold settings and PMT operating voltages.  This work involves small jobs in the student machine shop and soldering. This team of students has maintained the involvement of Lincoln area (and a few more distant) high schools during the years without major grant support.

CROP draws undergraduate students and through this award we would like to continue training them to work with area high schools, even without the workshops we have run under a fully funded program. We need to revive our own rooftop array to provide a continuous source of air shower data. We would like to package detectors, GPS antenna, and DAQcards into kits, supplemented with training videos, that individual students or high school student teams interested in science fair projects or individual research studies could check out.  We continue to be interested in providing educational experiences for the most remote rural students in the state. 

\paragraph{Elementary Teacher Physics Course}
Claes partners with faculty from the College of Education and Human Sciences on several projects working to improve K-12 science instructors knowledge and understanding of physical science content and concepts.  This includes his involvement as co-PI in two NSF Noyce grants. In Spring and Summer 2018 he created and taught a brand new course. ``Concepts in Physics for Elementary Teachers" providing genuine science content for participants of NebraskaSTEM, a five-year elementary STEM degree and professional development program for teachers funded through yet another NSF Noyce grant. This program provides a Master’s in rural STEM education (completed in the first year) along with follow-up engagement in STEM leadership activities, and participation in the nationwide community of Noyce Master Teaching Fellows. 
%Claes's course design was built on an argument that a good elementary teacher is not merely proficient in the content at the level it is to be taught, but having knowledge beyond that, providing a deeper understanding of the concepts, can critically inform the way it is taught.  
The idea behind Claes's course design is that a good elementary teacher should not be merely proficient in the content at the level it is to be taught. Knowledge beyond this level and deeper understanding can critically inform the way it is taught.
15 rural Nebraska elementary teachers took this 3-week inquiry-based course. Each day was divided into observational play with interactive demos, instructor-led inquiry into what was observed, periods of experimental design, hands-on construction of devices to be employed in the study, data collection, analysis, and peer instruction sessions designed to lead to the formation of Newton’s Laws and the Conservation of Energy and Momentum. All equipment was built with easily obtained pieces. Many of the activities were centered around foot long 2"$\times$4" blocks converted into student-built vehicles with skateboard wheels/bearings sets, spring bumpers, and velcro’ed ends. Launchers were built with PVC frames and elastic exercise bands and coil springs.  Speeds and accelerations was measured using the burst mode of students' cell phone cameras.  Claes will continue to develop and teach this course.

%XXX Dan and Greg: if you have anything in addition to the Speaker's Bureau, please add.
\paragraph{Other} In addition to the above, other ongoing efforts will continue.  Bloom will continue writing occasional pieces for Fermilab News, which has a wide audience inside and outside the community.  He has also joined the Quarknet Advisory Board as the CMS representative and will take part in charting the long-term direction for this successful long-term nationwide effort to support science teachers.  Bloom also continues to serve on the OSG Council, representing U.S.~CMS but working to ensure that all scientific communities can get beyond their desktop machines and have access to large high-throughput computing systems. Snow will  continue to serve on the FNAL's Users Executive Committee, which has broad impact within the U.S. community. Claes and Snow are members of UNL's Speakers Bureau~\cite{bib:speakers} and are routinely invited to give public talks.
