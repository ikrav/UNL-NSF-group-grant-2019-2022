\section{Broader Impacts of the Proposed Work}

%XXX Anyone: ok to mention underrepresented minorities below even though BEST is gone?
%XXX This paragraph is kept from the last proposal because it is good and does not need changes.
The UNL HEP group leads a number of activities that provide broader impacts beyond the intellectual merit already described. Some of these are unique within the field or have inspired similar efforts elsewhere. The outcomes of these programs include  greater participation of underrepresented minorities in STEM fields, increased public scientific literacy and public engagement with science and technology, and the improved well-being of individuals in society.

\paragraph{CMS Education and Outreach}
UNL postdocs have been making an outstanding contribution to international outreach
efforts at CERN as described in Section~\ref{sec:prior}.
Our postdocs and graduate students stationed at CERN will continue this work in the
next several years. Specific planned activities are:
%XXX This bullet list can be converted to plane text if space is an issue.
\begin{itemize}

     \item Conduct virtual tours at CERN.  UNL
   	postdocs will continue to serve as guides in the virtual tours offered by CERN
       Outreach. The tours involve international participants, usually high school
       students or the general public, video-conferencing with a tour guide at CERN.
       The guide visits facilities with a cameraman following her and gives
       explanations and answers questions in a real-time live video connection.
       %The virtual tours are open to anyone, and members of the general public from U.S. have already
       %taken advantage of them. %Each guide conducts several virtual tours per year.

     \item Conduct on-site facility tours at CERN for visiting members of the public. We
       expect our postdocs to continue serving as CERN-certified guides for CMS,
       other CERN experiments, and the accelerator facilities. The expected
       service is order of one facility tour per month (several per month during
       LHC shutdown periods). Many visitors from the U.S. including school
       children, dignitaries and the general public benefit from these tours.

   \item Our postdocs are expected to be participants of short outreach videos
       produced by CMS/CERN and U.S.~CMS/FNAL outreach, as they have been in the last
       several years.
\end{itemize}

%XXX Dan, please review commented out text below. Is it still relevant? Anything happened on that? Not relevant anymore?
% At UNL, our group is also working with the university's Sheldon Museum of Art to explore hosting an ``art@CMS"~\cite{bib:artcms} exhibition in the museum a year or more from now.

\noindent
%XXX TODO Greg: please check the paragraph below. It is meant to just highlight
%  the broader impact of this work, the actual description is in the "future" section.
Snow will continue his leadership in the CMS Thesis Award and the CMS Named Lectures programs already described in detail in Section~\ref{sec:prior}. A recognition through such an award or a named lecture enhances motivation of the best CMS graduate students and postdocs and adds to their CVs, helping to build a career they deserve.  Snow will also continue to serve in the FNAL's Users Executive Committee which has broad impact within the U.S. community and internationally through its education and public outreach program.

% XXX Ken, if you plan to involve undergrads in any computing projects, please add relevant text.
The group's faculty will provide UNL undergraduate students with a unique research experiences, involving them in the group's CMS projects. Among the expected activities are the contribution of physics majors to simple testing equipment software development, electronics circuit design, and construction of Phase~2 TFPX modules in the group's silicon lab, overseen by Kravchenko and Claes. The importance of undergraduate research cannot be  overemphasized: it helps retention rates, helps students become better future 
scientists and professionals, and tremendously increases their motivation.

%XXX Dan, please revise the CROP paragraph and your add other activities as separate paragraphs 
\paragraph{CROP}
Through an ITEST Strategies proposal (Action-at-a-Distance) CROP now partners with the National Center for Research on Rural Education (NCR2Ed) and (to date) a small selection of the Educational Service Units (ESUs) which cover the rural state of Nebraska. For years this project has engaged Lincoln/Omaha metropolitan high schools in a study of cosmic ray air showers. The current aim is to serve the needs of students in the most remote school districts of our state, and provide the means and incentives for the professional development of many instructors who are teaching out of their field, and study the efficacy of such efforts. 

By the latest census figures, nearly 50 million (17\% of the U.S. population) live in rural areas; an even larger total (38\%) live in small-town America.  Over 41\% of Nebraska's population is rural. Few outreach efforts reach rural communities. At best they either target small districts within commuting distance (hardly remote) or make single (even if extended) visits with no regular follow-up. Exploiting the connectivity offered to schools through their ESU we are exploring combinations of video-conferencing, video-streamed training, and a series of summer workshops and academic year conferences hosted through the ESU centers. Education research questions are addressed by comparing the experiences of remote participants with those of our local CROP schools, seeking to optimize the contact necessary to sustain interest and activity.

We enlist 5-6 undergraduates each year to build scintillator detectors, make efficiency measurements, and conduct coincidence studies of cosmic ray rates.  All are working on UNL’s rooftop array atop our physics building. Decentralized 3-day workshops hosted at ESU offices, coordinated by regional ESU staff and local CROP teachers, have replaced the 4-week summer workshops we previously held on campus. Each semester a Saturday research conference provides our high-school student and teacher teams a venue for reporting their work. An established a network of speakers from local tech industries (both urban and rural Nebraska) conduct successful and highly animated Q\&A sessions with students on STEM career opportunities at these conferences. An on-demand solution center is up and running to promptly respond to student queries, and all staff is being trained in the use of the CAMTASIA Suite, the software to be used in creating educational video training modules.

%XXX Ken, please describe your Quarknet plans. Ideally, we'd have 4-5 lines to warrant having a paragraph. A paragraph is good because it is prominent, with the activity name in bold. It is good to have something in addition to CROP.
% If this is not reasonably possible here, then please add the Quarknet somewhere else.
\paragraph{Quarknet}

%XXX Ken: please update your Quantum Diary part and the rural schools part.
%XXX Dan and Greg: if you have anything in addition to the Speaker's Bureau, please add.
\paragraph{Other} In addition to the above, other ongoing efforts will continue.  Bloom will extend his seven-year run as a regular blogger for the Quantum Diaries website.  The site, which is operated by interactions.org, the consortium of communications offices of the worlds particle physics laboratories, has an international readership.  He will also continue to visit under-served rural school districts in Nebraska as time and opportunity permit.  Claes and Snow are members of UNL’s Speakers Bureau~\cite{bib:speakers} and are routinely invited to give public talks. 
