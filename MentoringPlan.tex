\documentclass[11pt]{article}
\usepackage[margin=1in]{geometry}
\usepackage{proposal}
\begin{document}
\begin{center}
{\large \bf Postdoctoral Researcher Mentoring Plan}\\
\end{center}

Postdoctoral researchers are ultimately the life blood of any major
particle physics experiment.  As they are typically resident at the
laboratories where the experiments are actually conducted (in this case, at
CERN) while the PIs are not, the postdocs have the great responsibility of
successfully executing the projects.  Given the important tasks that we
have assigned to them, it is our responsibility to make sure that the
postdocs have received the best mentoring that we can provide.  We want to
make it possible for all of our postdocs to meet their own individual
career goals, and we will be as supportive as we can of their professional
development.  We consider ourselves to be successful in this task, as many
of our our postdocs have moved on to tenure-leading positions in particle
physics, or positions that will keep them on that path: Michael Eads is now
a full professor at Northern Illinois University; Helena Malbouisson, a
native of Brazil, on the faculty of UERJ in Rio de Janeiro; Jamila Bashir
Butt, of Pakistan, an assistant professor at COMSATS University in
Islamabad; Frank Meier will complete a habilitation degree at the
University of Heidelberg; and Rebeca Gonzalez Suarez recently joined the
faculty at Uppsala University in Sweden.  Rachel Bartek now holds a
research faculty position at Catholic University of America.  Other
postdocs are finding success in the private sector: Ioannis Katsanos now
works for the Commonwealth Edison utility in the Chicago area; Suvadeep
Bose is a data scientist with Discover Financial Services; and Jose
Lazo-Flores is a founding partner and director at NJ Analytics and
Applications in Zurich.

Postdoctoral mentoring in our group takes two forms.  First, we make sure
that the postdocs are making steady progress in their daily work.  This is
achieved through regular meetings to discuss status.  The UNL HEP group
meets every week, and all group members, including postdocs, give oral
reports.  Additional smaller meetings each week are focused on the
technical details of specific projects and data analyses.  Along the way we
advise the postdocs on how to carry out their work and how to present it to
others most effectively.  Faculty members are in daily contact with the
postdocs via email and Skype, and, when they visit CERN, they
make a point of spending time with the postdocs to catch up on any and all
issues.

Second, we make sure that the postdocs' activities are aligned with their
career goals.  This is most straightforward for postdocs who wish to build
careers in particle physics.  We ensure that their work is recognized
within the experimental collaborations, and we proactively promote them for
leadership positions on the experiments.  For instance, we strongly
supported Gonzalez Suarez for the leadership of the CMS top physics group.
We help postdocs increase their visibility in the broader particle-physics
community by arranging for them to present significant results at major
conferences of the field; funds are included in this proposal to allow the
postdocs to attend these meetings.  As an example, Stieger gave a
presentation this year at the ICHEP conference in Korea.  Once a year,
we hold a more formal meeting with each postdoc in the group to take stock
of their achievements and what they want to achieve in the coming year, and
we try provide each postdoc with regular opportunities to visit Lincoln.

As today's postdocs will be tomorrow's faculty members, it is important
that they too have opportunities to serve as mentors. When graduate
students are posted at CERN  for extended periods of time, the
postdocs resident there serve as their day-to-day supervisors. They
thus play a major role in shaping the students' research work, and often
provide critical support in helping the students to completion of their degrees.

In addition, UNL has an Office of Postdoctoral Studies which provides
professional and career development resources.  While many of our postdocs are
not resident at UNL and thus cannot take part in the Office's
on-campus programs, they are all aware of the available online resources
and the opportunity for a free membership in the National Postdoctoral
Association.  The Office provides extensive support to the mentors of
postdocs, and we plan to draw on their resources.


\end{document}
