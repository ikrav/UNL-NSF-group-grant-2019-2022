\documentclass[11pt]{article}
\usepackage[margin=1in]{geometry}
\usepackage{proposal}

\pagestyle{empty}

\begin{document}
\begin{center}
{\large \bf Project Summary}\\
\end{center}

{\Huge\color{red} OLD, NEEDS REWRITING}

{\bf Overview:} This proposal seeks continued support for fundamental research in particle 
physics at the energy frontier by the experimental high-energy
physics (HEP) group at the University of Nebraska-Lincoln (UNL) for the
three-year period starting 9/1/2016.  The group currently consists of five
faculty member PIs (K.~Bloom, D.~Claes, A.~Dominguez, I.~Kravchenko and
G.~Snow), three postdocs (two at CERN, one at UNL), five graduate students 
and six undergraduates.  Thanks to strong institutional support from UNL, 
this group has more than doubled in size and scientific impact over the 
past decade, and is poised for still greater achievements.

 The research effort will be focused on data analysis, operations, 
and the Phase~1 and Phase~2 upgrades for the Compact Muon
Solenoid (CMS) experiment at the Large Hadron Collider (LHC),
where UNL members group hold leadership roles in software and
computing (including hosting a Tier-2 center and management of the U.S. CMS operations program) and the upcoming
upgrade of the forward pixel detector.  The plan presented in
this proposal is well aligned with the primary scientific drivers
of the field established by the Particle Physics Project
Prioritization Panel.  In Run 2 of LHC at
13\TeV collision energy that started this year, CMS is poised to explore the highest
energy scales ever probed in accelerator-based experiments, and
is expected to observe new phenomena that could change our
understanding of the nature of matter and space. 

The UNL group has made major contributions to the output of the
CMS physics program during the Run~1 and post-Run~1 periods. The
group has a broad physics reach, with past and, as the Run~2 data
are already rolling in, ongoing efforts in measurements on Higgs,
top, electroweak, and B physics, plus searches for new phenomena.
In addition to physics measurements and operations, the contribution to CMS upgrades 
takes a central place in this proposal and includes production of 
critical components of the Phase~1 forward pixel detector,
a top managerial role in the whole U.S. CMS Phase~1 upgrade project, and
R\&D toward the Phase~2 CMS pixel detector. The planned work also
includes initiatives on offline computing of CMS, contributions to
electron and photon reconstruction and identification, detector operations,
and other activities vital to functioning of the CMS experiment.

{\bf Intellectual merit:} The LHC is one of the premier scientific
instruments of our time, and the observation of the Higgs boson
there has opened a new era of discovery and understanding in
particle physics. The new, highest-ever LHC energy and luminosity
will provide the most stringent tests of the standard model and will
lead to discoveries of -- or constraints on -- the new physics.
The UNL group's physics measurements are at the heart of the LHC
program, and tasks in computing, detector operations, and
construction of future detectors for the LHC are crucial for the
short- and long-term success of CMS. The work done in computing
for CMS has helped broaden the capabilities of distributed
high-throughput computing in general.

{\bf Broader impact:}
Through the unique NSF-funded Cosmic Ray
Observatory Project (CROP) outreach program, Claes and Snow have
established themselves as outreach specialists. The UNL group has
established a growing network of 29 high-school cosmic-ray
research teams, studying extensive air showers in Nebraska with
student-built and operated detectors, and future growth will
focus on under-served rural populations. The ongoing CROP
initiative has served as a model for similar high-school
cosmic-ray efforts across the U.S., Canada, and abroad. Among the
out- growths of CROP are Snow’s leadership of education and
outreach efforts at the Auger Observatory. Dominguez’s Bilingual
English Speaking Tutors (BEST) program engages 70 bilingual
elementary- school students and high-school tutors each year,
generating measurable improvements in student achievement, and a
significant expansion is planned. Smaller additional efforts
have had impacts on diverse populations in Lincoln and across
Nebraska. Members of the group participate in several LHC outreach 
programs at CERN as well as U.S. CMS outreach at Fermilab.
\end{document}


