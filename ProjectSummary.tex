\documentclass[11pt]{article}
\usepackage[margin=1in]{geometry}
\usepackage{proposal}

\pagestyle{empty}

\begin{document}
\begin{center}
{\large \bf Project Summary}\\
\end{center}

{\bf Overview:} This proposal seeks continued support for fundamental research in particle physics at the energy
frontier by the experimental high-energy physics (HEP) group at the University of Nebraska-Lincoln (UNL) for the
three-year period starting 8/1/2019.  The award would support the work of four faculty members in the group (K.~Bloom, D.~Claes, I.~Kravchenko and G.~Snow), who currently work with two postdocs (both at CERN) and five graduate students.  Thanks to strong institutional support from UNL, 
this group has more than doubled in size and scientific impact over the 
past fifteen years, and is poised for still greater achievements.

The research effort will be focused on data analysis, operations, and the Phase~2 upgrade for the Compact Muon Solenoid (CMS) experiment at the Large Hadron Collider (LHC).  Among the most visible activities of the group is hosting a Tier-2 computing center, management of the U.S.~CMS software and computing operations program, and construction of a significant portion of the Phase~2 forward pixel detector. The plan presented in this proposal is well aligned with the primary scientific drivers of the field established by the Particle Physics Project Prioritization Panel, as the CMS experiment continues to explore the highest energy scales ever probed in accelerator-based experiments, and is expected to observe new phenomena that could change our understanding of the nature of matter and space. 

The UNL group will continue making major contributions to the output of the CMS physics program as it has during Run~1 and Run~2. The group has a broad physics reach, with ongoing and planned measurements on Higgs, top, and electroweak physics, plus searches for new phenomena. The contribution to CMS upgrades takes the central place in this proposal through the preparation for and beginning of the construction of parts of the Phase~2 CMS Inner Tracker. The planned work also includes initiatives on offline computing, contributions to electron and photon reconstruction and identification, detector operations, and other activities vital to functioning of the CMS experiment.

{\bf Intellectual merit:} The LHC is one of the premier scientific
instruments of our time, and the observation of the Higgs boson
there has opened a new era of discovery and understanding in
particle physics. The comprehensive LHC dataset 
will allow stringent tests of the standard model and will
lead to discoveries of -- or constraints on -- new physics.
The UNL group's physics measurements are at the heart of the LHC
program, and tasks in computing, detector operations, and
construction of future detectors for the LHC are crucial for the
short- and long-term success of CMS.

{\bf Broader impact:}
Through the unique Cosmic Ray
Observatory Project (CROP) outreach program, Claes and Snow have established themselves as outreach specialists.  High school-based cosmic-ray research teams around Nebraska study
extensive air showers with student-built and operated detectors, and future growth will focus on under-served rural
populations. The ongoing CROP initiative has served as a model for similar high-school cosmic-ray efforts across the U.S., Canada, and abroad. A number of undergraduate students are expected to have strong involvement in the Phase~2 detector components construction as well as the CROP project, which will provide them a unique research experience. Smaller additional efforts
have had impacts on diverse populations in Lincoln and across
Nebraska. Members of the group participate in several LHC outreach programs at CERN and hold positions that allow them to guide the impact of particle physics innovations across the sciences.
\end{document}


