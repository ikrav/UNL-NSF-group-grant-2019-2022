\documentclass[11pt]{article} 
\usepackage[margin=1in]{geometry}
\usepackage{proposal}
\begin{document}
\begin{center}
{\Large \bf Budget Justification}\\
\end{center}

\section{Personnel} 

\paragraph{Faculty} PI Kravchenko (2 summer months effort/year) will be
responsible for overall project management and reporting to NSF.  Co-PI's
Bloom (1 summer month/year) and Claes (1 summer month/year) will also be supervising the research effort.  Because Claes
holds an administrative position at UNL, and Bloom is Software and
Computing Operations Manager for the U.S.~CMS Operations Program, they
request less support than the PI for this project.

\paragraph{Post-doctoral Research Associates} Three postdocs will work
full-time on the project.  Postdocs Finco and Stieger (both at CERN) have
been supported by the NSF award that is the predecessor of this proposal.
These postdocs are currently doing critical work on development of
reconstruction software, computing tools, and analysis of CMS data.  A new
postdoctoral researcher, to be resident either at UNL or FNAL, would be
hired to support research and development for the Phase~2 pixel detector,
which is becoming an area of critical need for the experiment.

\paragraph{Graduate Research Assistants} Four GRA's will work full time on
the project.  CMS students Kamalieddin, Fangmeier, Siado and Tabb make
significant contributions to CMS operations, hardware and software
development and data analysis.  We request continued support for them and
their successors after they graduate; our graduate-student recruiting
efforts have reached the point where an average of one student per incoming
class wishes to join the group.

\paragraph{Undergraduate research assistants} Four undergraduate
researchers will work on the project part time during the school year.  They
will work on well-defined well-defined hardware, software and data analysis
projects.  Undergraduate salaries are based on a 35-week academic year of ten-hour weeks and 15 weeks of 40-hour summer weeks, with a wage of \$10/hour.

\paragraph{}
\noindent A 2\% cost of living increase has been applied to all salaries in Years 2
and 3 of the project.  For the purposes of NSF limitations on salary compensation requested for senior personnel, a ``year" is defined by UNL as September 1 through August 31.

\section{Benefits}
Faculty and postdoc benefits are estimated at 30\% and 40\% of salary,
respectively.  GRA benefits include tuition remission estimated at 38\% of
salary and health benefits estimated at \$2,186 in Year 1; \$2,361 in Year
2; and \$2,550 in Year 3. The actual cost of benefits for each person will
be charged to the project.

\section{Equipment} 
None requested.

\section{Travel}
Domestic and foreign travel are critical to the project, as the bulk of the
actual experimentation takes place at CERN.  Many universities place at
CERN, where there is a critical mass of people working on the project and
the intellectual environment is very strong.  Fermilab is also an important
locus of U.S. activity, as the home of the U.S.~CMS Operations Program and
the LHC Physics Center (LPC).  Faculty, postdocs and students will need to
visit FNAL and CERN regularly to attend collaboration meetings, take shifts
operating the experiment, and interact with UNL staff and other
collaborators resident at these labs.  Travel expenses are reimbursed
through the University of Nebraska.

We anticipate that the three faculty members will travel to CERN a
total of four times a year for CMS meetings, and six times a
year to FNAL to visit the LPC and U.S.~CMS program leaders.  In addition,
U.S.-based postdocs and student will visit CERN on occasion, and CERN-based
postdocs and students will visit the U.S. now and then.

We also request funds for members of the group to attend HEP conferences.
Conference travel is essential for the dissemination of results and to keep
group members (especially postdocs and students) informed of the latest
developments in the field and visible to potential future employers.  We
anticipate that the ten members of the group will attend a total of
six domestic conferences and four foreign conferences per year.  Estimated
costs for travel in the first year are detailed in Table~\ref{tab:trips}.



\begin{table}[h]
\centering
\begin{tabular}{|l|cc|c|}\hline
Trip &  Trips/year & Cost/trip (\$) & Total\\\hline
PI to FNAL & 6 & 600 & 3600\\ 
Domestic conference & 6 & 1500 & 9000\\\hline
Domestic travel & & & 12600\\\hline
PI to CERN & 4 & 2500 & 10000\\
US postdoc/student to CERN & 3 & 2500 &7500\\
CERN postdoc/student to UNL/FNAL & 1 & 2500 & 2500\\
Foreign conference & 4 & 3000 & 12000\\\hline 
Foreign travel & & & 32000\\\hline
\end{tabular}
\caption{Estimated rate and cost of travel for the first year.} 
\label{tab:trips}
\end{table}



\section{Other direct costs}

\paragraph{Supplies} All personnel involved in the project will
require computers (laptop or desktop) to carry out their research; they are
necessary for access to data and for the control of experimental equipment.
CERN and FNAL do not provide computers to visiting researchers from
universities, and thus every member of the group needs their own computer
to make use of the other research infrastructure provided by these
laboratories.  

We expect that we will need to purchase about six systems for the ten
people supported by the grant during the three years of the proposed grant,
or about two systems each year.  The typical cost of a replacement
computer is \$1500.

\$1000/year is budgeted for consumables such as printer supplies, paper,
tools, and replacement parts and repairs for lab and computer equipment no
longer eligible for coverage under maintenance contracts.  Another \$1500
is budgeted for supplies to support the operation of the equipment in the
pixel development laboratory.

\paragraph{Publication costs} \$200/year is proposed for publication and
documentation, to cover costs of in-house duplicating and outside printing.

\paragraph{Other: Subsistence costs} Our commitments on CMS require us to
maintain a full-time presence at CERN, for the operation of the experiment
and so that students and postdocs can interact with a wide range of
collaborators.  Because of the significant cost of living in Switzerland
compared to the United States, we must provide a cost of living adjustment
to our Switzerland-based postdocs and students.  This is set to the 1500
CHF/month for postdocs and 1280 CHF/month for students that is used by the
U.S.~CMS Operations Program.  An exchange rate of 0.9 CHF/\$ is assumed.
We expect to have two postdocs stationed at CERN throughout the award
period.  An average of 1.5 graduate students will be resident at CERN
throughout the award period.

We anticipate that faculty members will spend significant amounts of time
at CERN during summers and for occasional stays during the school year to
help oversee CMS operations and shutdown activities; this integrates to three
months of faculty presence at CERN per year, for which we include a cost of
living adjustment of 3000 CHF/month.

\section{Indirect cost calculation} The preponderance of activity supported
by this proposal takes place away from the UNL campus. UNL's federally
negotiated rate is 26\% of MTDC for off-campus organized research.  MTDC base and F\&A costs are shown in Table~\ref{tab:MTDC}.

\begin{table}[h]
\centering
\begin{tabular}{ccc}\hline
 & MTDC base  & F\&A costs \\\hline
Y1 & \$515,214 & \$133,956 \\ 
Y2 & \$523,521 & \$136,115 \\ 
Y3 & \$532,043 & \$138,331 \\ \hline
\end{tabular}
\caption{MTDC base and F\&A costs.} 
\label{tab:MTDC}
\end{table}



\end{document}
