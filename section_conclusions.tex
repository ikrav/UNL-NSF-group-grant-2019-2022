\section{Summary}

The UNL HEP group is poised for major research accomplishments in 2019-2022 as the CMS experiment completes Run-2 measurements, starts operations in Run-3 at 14~TeV, and prepares for HL-LHC detector upgrades. The UNL HEP faculty members, with a complement of outstanding postdocs and graduate students, will be focused on the CMS experiment through this period.

The group pursues a wide spectrum of high-profile physics measurements that include detailed measurements of Higgs boson properties, top physics, standard model precision measurements, and searches for new physics. With our interests this broad, we have excellent prospects of making discoveries no matter where they show up in data. The years covered by this proposal is the time when CMS will publish most interesting results based on the full Run-2 data set of CMS, with the UNL HEP group leading analysis teams that produce many of them. A critical activity of our group for the future of CMS during the HL-LHC is R\&D, preparation for, and the start of module production at UNL for the CMS Inner Tracker. The group plans to significantly contribute to CMS reconstruction revisions and software development in preparation of Run-3. The CMS computing and data operations, where our group is also heavily invested, will be critical for making possible the Run-2 legacy measurements in 2019-2020 and for making data collection and early data analyses seamless in Run-3 in 2021-2022. This broad and balanced program of work will not only give us access to the most important physics but also provide complete education and training to our students and postdocs.

The UNL HEP group has generated this level of activity by consciously leveraging our previous support to provide new opportunities. While we work on a broad set of topics and projects within CMS, we remain a single coherent research group that now includes a junior faculty member, and our research topics well overlap with a small HEP theory group at UNL. The group's activities are synergistic and group members working on different topics remain in close contact.  We foster an environment of intellectual activity and excitement in our work, and we encourage our younger colleagues to fulfill their potential as physicists. With that environment in place, and our recent history of growth and success, we are ready and eager to carry out this program.
