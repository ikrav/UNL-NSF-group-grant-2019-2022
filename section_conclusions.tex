\section{Summary}

The UNL HEP group is poised for major research accomplishments in 2019-2022 as the CMS experiment completes Run~2 measurements, starts operations in Run~3 at 14~TeV, and prepares for HL-LHC detector upgrades. The UNL HEP faculty members, with a complement of outstanding postdocs and graduate students, will be focused on the CMS experiment through this period.

The group pursues a wide spectrum of high-profile physics measurements that include detailed measurements of Higgs boson properties, top physics, standard model precision measurements, and searches for new physics. With these broad interests, we have excellent prospects of making discoveries no matter where they show up in data. The period covered by this proposal is the time when CMS will publish the most interesting results based on the full Run~2 data set of CMS, with the UNL HEP group leading multiple analysis teams. A critical activity of our group for the future of CMS is R\&D, preparation for, and the start of module production at UNL for the HL-LHC upgrade of the Inner Tracker. The group will contribute significantly to CMS reconstruction revisions and software development in preparation of Run~3. CMS computing operations and development, where our group is also heavily invested, will be critical for Run~2 legacy measurements in 2019-2020 for making data collection and early data analyses seamless in Run-3 in 2021-2022, and preparing for the HL-LHC. This broad and balanced program of work will not only give us access to the most important physics but also provide complete education and training to our students and postdocs.

The UNL HEP group has generated this level of activity by actively leveraging our previous support to provide new opportunities. While we work on a broad set of topics and projects within CMS, we remain a single coherent group with synergistic research topics that also interacts with our new faculty colleagues.  We foster an environment of intellectual activity and excitement in our work, and we encourage our younger colleagues to fulfill their potential as physicists. With that environment in place, and our history of growth and success, we are ready and eager to carry out this program.
