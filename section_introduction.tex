\noindent
%XXX Is P5/science drivers still the latest and best guidance from funding
%XXX agencies to refer to?
The Large Hadron Collider is finishing its Run-2 era of operation at 13~TeV. The Higgs boson, discovered at the LHC during Run-1, is now firmly established with many of its basic properties accurately measured. Numerous precision measurements and first observations are performed in processes involving top quarks, electroweak bosons, and others. All throughout these experimental advances the standard model (SM) of particle physics stands strong, with all predictions confirmed and no significant deviations found to date. LHC experiments have probed for new physics predicted by a variety of proposed theories and excluded existence of new particles or effects up to multi-TeV scales. Experiments at the energy frontier at LHC, with the energy increase nearly 7-fold as compared to the Tevatron, have not resulted in direct observation of new phenomena so far. The most cost-effective strategy for the High Energy Physics (HEP) field at this point, adopted by the international community, is to invest in high luminosity machines and large data sets, with which rare processes of the SM and, possibly, new physics can be observed and detailed studies of Higgs, top quark, and electroweak bosons can be performed. The highest priority of the United States HEP program, as established by the Particle Physics Project Prioritization Panel (P5), remains to exploit the opportunities of the LHC and its planned upgrades~\cite{bib:P5}.

%XXX below check years, and maybe add specific integrated luminosity figures
The plan for the LHC operation as well as for the LHC experiments reflects this strategy, with the Run-3 at the the same collision energy being planned for the years 2021-2023, that would result in a significantly larger data set than that of the Run-2, followed by the High Luminosity LHC (HL-LHC) period that will start around 2025 and may continue for a decade with the luminosity and data set sizes increased by about x10.

 The University of Nebraska-Lincoln (UNL) HEP group is making significant, broad contributions to the Compact Muon Solenoid (CMS) experiment at the LHC through physics measurements, detector and computing operations, and detector upgrades. The group consists of five faculty members, two postdocs and five graduate students. This proposal is seeking funding for four of the faculty members: Bloom, Claes, Kravchenko, and Snow, while recently hired Golf is presently pursuing a CAREER award. The whole UNL HEP experimental group functions as a well coordinated team with all members focusing on the proton-proton program at CMS, with diverse physics interests and leadership in projects critical for the present and future operation of the experiment. Impressed by the demonstrated success of our group over the last 15-20 years in published high-profile results, attracted funding from NSF and DOE, developed on-campus computing and laboratory facilities, and overall visibility of the group, UNL administration has recently extended support for HEP research and a new faculty position was created. Last year, we hired theorist-phenomenologist Huang who joined our HEP community along with a postdoc and graduate student, working on topics that largerly overlap with the measurements done in the experimental group. 

Over the next three years, the group plans to maintain an active research program that engages with all aspects of the CMS experiment:

\begin{packed_enum}
\item We will perform physics measurements that explore the Higgs and top-quark sectors, test the standard model to great precision, and search for new physics.
\item We will take major part in the Phase~2 CMS upgrade by participating in the silicon pixel detectors R\&D, setting up and starting module production for the Inner Tracker of CMS.
\item We will contribute to the successful operation of the experiment through efforts in computing and object reconstruction and identification, using this work to improve our physics analyses.
\item We will hold leadership positions in many of these activities, giving us strong influence over the success of the experiment.
\end{packed_enum}

In addition, we will continue our signature  education and outreach activity: Cosmic Ray Observatory Project (CROP), a unique outreach effort to study extensive cosmic-ray air showers using high-school based detectors.
Together, these activities give the UNL HEP group a tremendous opportunity to drive the science of the CMS experiment.
