\noindent
%Ken's attempt at the introduction; don't worry, ultimately we will only keep one of these!
The field of experimental particle physics is in one of its most productive periods as the Large Hadron Collider (LHC) operates at the highest-ever collision energies, with the Run~2 period completed and preparation for future runs and  luminosity upgrades underway.
%
%Particle physics at the energy frontier remains scientifically compelling, with the opportunity to provide our first glimpse of physics beyond the standard model.  
So far, experiments at the LHC have confirmed the standard model (SM), including the existence of a Higgs boson.  However, studies of Higgs boson properties have just begun, and searches for additional particles have yet to yield any observations. There is great opportunity in the huge datasets provided by Run~2 to explore Higgs properties in detail, make precise tests of the SM, and extend searches for new phenomena.  In addition, the High Luminosity LHC (HL-LHC) will begin operations in 2026, and an order of magnitude increase in luminosity will soon provide even larger datasets for study.  This is why the Particle Physics Project Prioritization Panel has set the exploitation of the LHC and its planned upgrades as the highest priority of the U.S. particle physics program, with the goal of addressing the principal science drivers of using the Higgs boson as a tool for discovery, exploring the unknown through searches for new particles, interactions, and physical principles; and, if we are lucky, identifying the physics of dark matter~\cite{bib:P5}.

The University of Nebraska-Lincoln (UNL) HEP group is making significant, broad contributions to the Compact Muon Solenoid (CMS) experiment at the LHC through physics measurements, detector and computing operations, and detector upgrades. The CMS group consists of five faculty members, two postdocs and five graduate students. Members of the group all have CMS as their highest-priority research activity, have diverse physics interests and are leading projects critical for the present and future operation of the experiment.  The successes of the group over the past 15 years -- publication of high-profile results, funding from NSF and DOE, and development of on-campus computing and laboratory facilities -- has led to the university's increased support for HEP research, resulting in two faculty hires last year: CMS member Frank Golf, who is seeking NSF funding separately from the three PIs of this proposal to obtain his first reviews as an independent PI, and theorist-phenomenologist Peisi Huang, who works on topics that largely overlap with the measurements done in the experimental group and has recently been funded by NSF.

Over the next three years, the group will maintain an active research program that engages with all aspects of the CMS experiment:

\vspace{-0.2cm}
\begin{packed_enum}
\item We will perform physics measurements that explore the Higgs and top-quark sectors, test the standard model to great precision, and search for new physics with the Run~2 data set.
\item We will take a major role in the Phase~2 CMS upgrade by participating in silicon pixel detector R\&D, setting up and starting module production for the Inner Tracker of CMS.
\item We will contribute to the successful operation of the experiment through efforts in computing and object reconstruction and identification, using this work to improve our physics analyses and commission the detector for Run~3 starting in 2021.
\item We will hold leadership positions in many of these activities, giving us strong influence over the success of the experiment.
\end{packed_enum}
\vspace{-0.2cm}

In addition, we will continue our signature education and outreach activity, the Cosmic Ray Observatory Project (CROP), a unique outreach effort to study extensive cosmic-ray air showers using high-school based detectors, along with other activities that provide broader impact.  All of these activities give the UNL HEP group a tremendous opportunity to drive the science of the CMS experiment and share the excitement of this science with the community.
