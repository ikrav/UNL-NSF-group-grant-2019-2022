\documentclass[11pt]{article}
\usepackage[margin=1in]{geometry}
\usepackage{proposal}
\begin{document}
\begin{center}
{\large \bf Data Management Plan}\\
\end{center}

The projects described in this proposal do not produce any data on their own; they merely make use of data generated by the CMS experiment.  CMS has a data management policy, which is described at \url{http://opendata.cern.ch/record/411}.  This policy conforms with that of the Data Preservation in High Energy Physics (DPHEP) study group, which has described a hierarchy for the types of data that particle-physics experiments produce, and given recommendations for how such data should be preserved for future use.  The CMS policy also describes a plan for releasing data to the public.

The types of data the CMS produces include the raw data produced by the detectors, the reconstructed version of the raw data, and simulated events.  CMS has its own mechanisms for archiving data, and any data that resides at our institution will merely be a copy of data that is stored permanently by the experiment.  Thus, there is no need for this project to separately manage or archive any experimental data.  This explicitly includes data resident at the  Tier-2 computing center that we operate.  All data at the Tier-2 is archived on tape at at least one of the seven Tier-1 centers operated by CMS.

The analysis of the experimental data is described in published, peer-reviewed journal articles; summaries of data analyses that are released to the public (often as contributions to  conferences); and notes that are circulated internally within CMS. The journal articles are archived by the journals themselves, and are also typically available through the arxiv.org e-print archive. The public analysis summaries and internal notes are archived by CMS and are available through Web interfaces. 

\end{document}