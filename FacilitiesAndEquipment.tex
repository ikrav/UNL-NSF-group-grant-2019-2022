\documentclass[11pt]{article}
\usepackage[margin=1in]{geometry}
\usepackage{proposal}
\begin{document}
\begin{center}
{\large \bf Facilities, Equipment and Other Resources}\\
\end{center}

%%% - ok now? {\Huge\color{red} OLD, NEEDS REWRITING}

\section{Facilities}

\paragraph{Laboratory} The UNL HEP group has a 500 square-foot clean room devoted to assembly and testing of silicon-based particle detectors. The clean room is presently equipped with an F\&K Delvotec fully automatic wirebonder (model G5 64000). Other available equipment includes two vacuum pumps, optical table for precision assembly, high-end fast oscilloscope, as well as a standard array of smaller, precision lab tools. Additional equipment necessary to execute the silicon module production program of this proposal, including most notably a computer-controlled Aerotech gantry for module assembly, a probe station, and a test stand for module testing, is planned for UNL by U.S.~CMS upgrade management as part of the project budget for the Phase~2 upgrade, which has passed NSF Conceptual and Preliminary Design Reviews and is presently approaching the Final Design Review.

Testbeam facilities at Fermi National Accelerator Laboratory in Batavia, IL will be used for module testing.

\paragraph{Clinical} None.

\paragraph{Animal} None.

\paragraph{Computer} The US CMS Tier-2 computing cluster is part of UNL's
Holland Computing Center (HCC).  The cluster currently has more than 9500 batch slots and can host about 4.7~PB of CMS data.
HCC also operates the Tusker, Crane and Sandhills clusters, each of which has several thousand
processing cores; these resources are available to the HEP group on an
opportunistic basis. Along with the access to the local CMS Tier-2 cluster, UNL HEP group members have access to other massive computing facilities
of the CMS experiment. These include grid computing facilities distributed
across CMS collaborating institutions around the world and the CMS
computing resources available locally at CERN.  All group members have access to a 
desktop or laptop machine for personal computing.

\paragraph{Office} All faculty members have individual offices in Jorgensen
Hall, the home of the UNL Department of Physics and Astronomy, and on-site
graduate students have a shared office.  The group also has offices for
students, postdocs and visiting faculty at Fermilab at the LPC and at CERN (in Building 8).

\paragraph{Other} Research is conducted at CERN in Geneva, Switzerland
(site of the CMS Experiment) and at the LHC Physics Center at Fermi National Accelerator Laboratory in Batavia, IL. The UNL HEP group also owns one Polycom
videoconferencing unit and has access to another one that is owned by the
Department of Physics and Astronomy.

\section{Major Equipment}

The project makes use of the CMS detector at CERN, a comprehensive,
multi-element nearly $4\pi$ detector for studying hadron collisions.  

\section{Other Resources}

Secretarial support and business-office support are provided by the
University of Nebraska-Lincoln Department of Physics and Astronomy. The department hosts excellent instrument and electronics shops, with full-time permanent staff members, which are available for HEP group projects at reduced in-house hourly labor rates.

\end{document}