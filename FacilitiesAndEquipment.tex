\documentclass[11pt]{article}
\usepackage[margin=1in]{geometry}
\usepackage{proposal}
\begin{document}
\begin{center}
{\large \bf Facilities, Equipment and Other Resources}\\
\end{center}

{\Huge\color{red} OLD, NEEDS REWRITING}

\section{Facilities}

\paragraph{Laboratory} The UNL HEP group has a 500 square-foot clean room for pixel detector assembly and testing.  It is equipped with a computer-controlled Aerotech gantry that is used for precision pick-and-place assembly using custom designed tooling; a manual gold-ball/aluminum-wedge wire bonder; an automatic aluminum-wedge wire bonder with pull tester; test stands that can be used to operate and study the CMS pixel modules from room temperature down to $-25^{\circ}$C; a probe station with micropositioners and Picoprobes which is used to test pixel detector sensors and other ASICs; and assorted equipment such as oscilloscopes, microscopes and power supplies.  Additional space and electronics equipment are available for the construction and repair of scintillating detectors for CROP. A high-bay
area is also available for the construction of larger components of
particle-physics experiments.

The UNL HEP group can use testbeam facilities available in the Extreme Light Laboratory situated on the UNL campus for pixel detector development and performance studies. PIs 
of this proposal Claes and Dominguez are members of the Extreme Light Research Core Facility Executive Committee. The laboratory's centerpiece Diocles laser has a combined peak power of 140~TW, a pulse width of 25 fs, and a repetition rate of 10 Hz, which makes it the highest average power laser in the U.S. This laser can provide tunable monoenergetic laser-driven synchrotron x-rays, electron beams accelerated by plasma wakefields, in addition to the pulsed laser light. A second laser, Archimedes, operating at higher repetition rates, expands the range of experiments and applications of the lab. Beamlines can be delivered through vacuum pipes to three separate user stations, so several experiments can be run simultaneously.

Dominguez and graduate student Fangmeier are completing the design and construction of a new test beam telescope which will be used to study the performance of novel silicon pixel detectors in beam lines, such as the wakefield accelerated electrons from the Extreme Light lasers at UNL.  This telescope is expected to be ready by the end of 2016.

\paragraph{Clinical} None.

\paragraph{Animal} None.

\paragraph{Computer} The US CMS Tier-2 computing cluster is part of UNL's
Holland Computing Center (HCC).  The cluster currently has more than 5000 batch slots and can host about 2.0~PB of CMS data.
HCC also operates the Tusker, Crane and Sandhills cluster, each of which has several thousand
processing cores; these resources are available to the HEP group on an
opportunistic basis. Along with the access to the local CMS Tier-2 cluster,
the UNL HEP group members have access to other massive computing facilities
of the CMS experiment. These include grid computing facilities distributed
across CMS collaborating institutions around the world and the CMS
computing resources available locally at CERN. At CERN, UNL also maintains two
desktop computers, unlhp and unlmac01. All group members have access to a 
desktop or laptop machine for personal computing.

\paragraph{Office} All faculty members have individual offices in Jorgensen
Hall, the home of the UNL Department of Physics and Astronomy, and on-site
graduate students have a shared office.  The group also has offices for
students, postdocs and visiting faculty at Fermilab at the LPC and at CERN (in Building 8).

\paragraph{Other} Research is conducted at CERN in Geneva, Switzerland
(site of the CMS Experiment) and at the LHC Physics Center at Fermi National Accelerator Laboratory in Batavia, IL. The UNL HEP group also owns one Polycom
videoconferencing unit and has access to another one that is owned by the
Department of Physics and Astronomy.

\section{Major Equipment}

The project makes use of the CMS detector at CERN, a comprehensive,
multi-element $4\pi$ detectors for studying hadron collisions.  

\section{Other Resources}

Secretarial support and business-office support are provided by the
University of Nebraska-Lincoln Department of Physics and Astronomy.  Administrative Assistant Christopher Hamilton is supported by PHY-1343486 to help administer the large Cooperative Agreement for the CMS Phase-1 Upgrades.  The
department hosts excellent instrument and electronics shops, with full-time
permanent staff members, which are available for projects of the high-energy
physics group at reduced in-house hourly labor rates.

\end{document}