\section{Results from Prior NSF Support}
\label{sec:prior}

%%% XXX Everyone, please check if all of your grants are listed.
%%% XXX Greg: any Auger-only NSF support?
%%%
In the past five years the research activities of the UNL HEP group have been supported by these NSF grants: 
% CMS ops grant, 2012-2016
PHY-1120138, ``U.S. CMS Operations at the LHC,'' PI Marlow (Princeton), \$43,794,718, 1/1/12-12/31/16; 
% CMS ops grant, 2012-2016
PHY-1624356, ``U.S. CMS Operations at the LHC,'' PI Marlow (Princeton), \$26,249,987 to date, 1/1/17-12/31/21; 
% Pixels, 2007-2014
OISE-0730173, ``PIRE: Collaborative research with the Paul Scherrer Institute and Eidgenoessische Technische Hochschule on Advanced Pixel Silicon Detectors for the CMS detector,'' PIs Bean (KU), Dominguez {\it et al.}, \$2,677,658, 10/1/07-9/30/14; 
% Computing AAA, 2011-2016
PHY-1104664, ``Collaborative Research: Any Data, Anytime, Anywhere,'' PI Bloom, \$710,336, 9/15/11-2/29/16;
% DASPOS
PHY-1247316, ``Data and Software Preservation for Open Science," PI Hildreth (UND) {\it et al.}, \$2,065,046, 9/15/12-8/31/17;
% Pixels upgrade, 2014-2019
PHY-1343486, ``U.S. CMS Phase-1 Upgrades,'' PIs Bean (KU), Dominguez {\it et al.}, \$9,355,988, 6/15/14-5/31/19;
% UNL HEP group grant 2013-2016
PHY-1306040, ``Experimental Particle Physics at the Energy and Cosmic Frontiers,'' PIs Bloom, Claes, Dominguez, Kravchenko and Snow, \$2,055,000, 9/1/13-8/31/16;
% CROP grant 2013-2016
DRL-1311782, ``Strategies: Action at a Distance ITEST,'' PIs: Claes, Snow, {\it et al.}, \$550,000, 08/01/13-07/31/16;
% ARA 
PHY-1404212, ``Collaborative Research: 2016-2019 Development of the Askaryan Radio Array Ultra-High Energy Neutrino Detector at the South Pole,'' PI Karle (UW), \$554,296, 01/01/16-12/31/18;
% Current HEP base grant 2016-19
PHY-1607202, ``Particle Physics Research with the CMS Experiment at the LHC," PIs Kravchenko, Bloom, Claes, Dominguez and Snow, \$2,070,000, 8/1/16-7/31/19.
(All award amounts are those of the total award; where non-UNL PI's are named, UNL received a sub-award.)


\subsection{Intellectual Merit}

\paragraph{CMS}
The UNL HEP group has made significant contributions to the CMS experiment across a broad range of activities: physics measurements, construction, operations, and preparations for upgrades. The activities completed span a wide array of topics for a mid-sized university group and have made a broad impact on the CMS research program.

{\bf Top production and properties} 
UNL group members have held multiple leadership positions in the CMS top
physics group.  Gonzalez Suarez was co-convener of the entire CMS top group
in 2016-18, after having previously served as co-convener of the single top
subgroup in 2012-14.  She was also one of the primary authors on several
measurements in single-top physics, including the analyses of the single
top production in the s- and t-channels, searches for flavor changing
neutral currents in events with single top, and
others~\cite{bib:single-top-papers}.  Stieger was co-convener of the top
mass subgroup in 2013-15, and shepherded several measurements from the
Run~1 data to completion.  He also led a top-mass measurement based only on
variable measured with the CMS tracker~\cite{bib:tracker-top-mass}.

%XXX Ilya: review and add references
{\bf Higgs production and properties}
After the Higgs boson discovery, the group pursued a number of studies of the Higgs boson in CMS. Kravchenko and Gonzalez Suarez were the primary authors of a search for Higgs production associated with a $Z$ boson and the subsequent decay of the $ZH$ system into the $3\ell+2\text{jets}+1\nu$ final state, the results of which were published in the CMS legacy paper on $H\rightarrow WW$ studies of Run~1~\cite{bib:HWWlegacy}. Kravchenko and Gonzalez Suarez have also completed a thorough investigation of different spin-parity hypotheses for the newly discovered boson with $H\rightarrow VV$, where the UNL group in collaboration with MIT covered the $H\rightarrow WW$ half of the paper~\cite{bib:higgs-spin-parity}. Gonzalez Suarez has served as a co-leader of the CMS $H\to WW$ working group in 2015-2016 and was a main author on several Higgs papers including the measurement of the Higgs boson transverse momentum [ref-HWW-pT] and the Higgs boson width [ref-HWW-widh].

{\bf Intersections of top and Higgs physics} Members of the UNL group have
been leaders in studies of the interactions between the top and Higgs
bosons.  This work started in Run~1, when Bloom and graduate student Daniel
Knowlton performed a search for the anomalous production of a single top
quark in association with a Higgs boson ($tHq$), using the $H \to b\bar{b}$
decay mode~\cite{bib:tHqbbPAS}.  They developed a data-driven method to
estimate the $t\bar{t}$ background that dominates the event sample.  At the
same time, before joining the UNL group, Stieger worked on a similar search
targeting the $H \to WW$ mode, with multilepton final states.  Bloom
co-edited the paper that combined those two results with two others that
used different Higgs decay modes~\cite{bib:tHqRun1}.  After Stieger joined
the group, he and Bloom worked with Monroy on the successor analysis, a
search for $tHq$ in multilepton final states using the 2016 CMS dataset.
This work was the basis for Monroy's PhD thesis~\cite{bib:monroy_thesis}.  A preliminary version of
this measurement, and its combination with measurements in the same dataset
using the $H\to b\bar{b}$ and $H \to \gamma\gamma$ decay modes, was
released for the 2018 ICHEP conference~\cite{bib:tHqRun2PAS} and submitted
for publication in XXXX 2018~\cite{bib:tHqRun2}.  This result constrains
the value of the top-Higgs Yukawa coupling and favors a positive value over
a negative value by about 1.5 standard deviations.

Stieger has also worked the other key measurement that probes the top-Higgs
coupling, namely the search for $t\bar{t}H$ production.  He contributed to
this search in the multilepton final state, leading to the first limits set
on this process in the Run~2 dataset~\cite{bib:ttHmultilep}.

{\bf Standard model precision measurements} Kravchenko with graduate students Kamalieddin and Avdeeva performed a number of precision measurements of Drell-Yan production at LHC at 7 and 8 TeV. Several published papers present single and double differential production cross sections with respect to mass and rapidity [add-DY-paper-refs].

{\bf New physics searches}
Claes and postdoc Suvadeep Bose working with several other collaborators have performed the first search for lepton-flavor violating Higgs decays in CMS. This direct search was conducted for the $H\rightarrow \tau\mu$ signature, and is an order of magnitude more sensitive than prior indirect searches. No lepton-flavor violating Higgs decays were found in the analysis and limits on the branching ratio for this final state have been set, as reported in the resulting paper~\cite{bib:higgs-LFV}.

Claes and Bose pursued searches for new physics looking for quark compositness, quark contact interactions, and extra spatial dimensions in CMS dijet events employing an angular analysis technique. No deviations from the standard model have been found and limits on the related new physics parameters were published~\cite{bib:quark-compositness-etc}.

Postdoc Rachel Bartek completed an analysis with the in 8\TeV data set looking for the production of excited bottom quarks, an exotic form of matter predicted by many theories beyond the standard model~\cite{bib:bstar}. 

{\bf Papers to be submitted soon} Several measurements where group members are among the primary authors are close to completion and presently going through the approval process in CMS. These include (only the closest to publication efforts are listed): a resonant di-Higgs search for new spin-0 and spin-2 particles in the channel $HH\to bbZZ\to bb\ell\ell\nu\nu$, the legacy Run-2 measurement of the SM $H\to\gamma\gamma$ process, a search for low mass non-SM Higgs boson decaying to a pair of photons, and a differential cross section of Drell-Yan production at 13~TeV. 

{\bf Computing} Since January 2015, Bloom has served as the Software and
Computing Operations Manager for the U.S.~CMS Operations Program, after
five years as Deputy Manager.  He is responsible for the entire \$16M
annual budget for software and computing, with approximately 55 FTE
supported by the project, and handles interactions with the funding
agencies and the institutes that carry out the work.  Most importantly, he
has set the strategic direction for the program, which he has pointed ever
more strongly towards the necessary preparations for the HL-LHC.  As part
of the latter, he developed tools to help project future computing resource
needs~\cite{bib:resource-modeling}; the results obtained from these tools
have been adopted by CMS as the official resource estimates that are
routinely shown to funding agencies and other external audiences, and have
become the basis for future resource estimations from the Worldwide LHC
Computing Grid (WLCG).

Bloom has overseen the operation of UNL's Tier-2 computing center, along
with David Swanson of UNL's Holland Computing Center (HCC). This center is
one of seven in the US and fifty in all of CMS. The center is supported by
two full-time staff members, and currently operates about 9500 batch slots
and hosts 4.7~PB of data. The Nebraska center is routinely ranked among the
best in all of CMS in just about every metric, and also has typically been
the pathfinder site for modernization of operations, through activities
like adoption of the Singularity containerization tool. Bloom also served
as the U.S.~CMS Level~2 manager for the Tier-2 program until becoming the
Level~1 manager for Software and Computing, and he has continued to be a
CMS-wide co-coordinator for all Tier-2 centers.

Bloom has also worked with HCC staff on other computing projects for CMS
that have had significant reach beyond particle physics.  The best known is
the NSF-funded ``Any Data, Anytime, Anywhere'' project, which has made data
access over the wide-area network transparent and efficient through the
creation of a world-wide data federation for CMS~\cite{bib:AAA}.  This has led to a
broader range of computing resources such as commercial clouds,
opportunistic sites on the Open Science Grid (OSG), and federally-funded
high-performance computing (HPC) centers that can be elastically included
in the CMS infrastruture~\cite{bib:hepcloud}.  It has also helped make CMS
computing operations more efficient by enabling new workflows such as the
addition of pre-mixed pileup events to simulated events, which has reduced
both processing time and I/O load on computing systems.

Under Bloom's supervision, Stieger and computing professional Alan Malta
Rodrigues (a computing professional funded by the U.S.~CMS Operations
Program and hosted by UNL) have developed improved monitoring systems for
CMS computing. Stieger has completed work on a new monitoring framework for
all CMS jobs being run on the distributed computing infrastructure, which
makes use of modern tools such as the ElasticSearch data analytics system
and the Grafana data visualization platform. Malta Rodrigues and Stieger
have developed the monitoring framework for the production workflow
management system (for which Malta Rodrigues is a key developer) to help in
system debugging.  Both monitoring frameworks are among the first to be
built on the CERN MONIT system that is becoming the baseline for all LHC
computing monitoring projects.

UNL has also hosted various U.S.~CMS Software and Computing Staff, such as
Sudhir Malik, now a faculty member at University of Puerto
Rico-Mayag{\"u}ez who led CMS computing user support and has grown into a
national leader in HEP computing education and training; computing
professional Rob Snihur, who co-led U.S. CMS Tier-3 facility support; and
currently computing professional Marian Zvada, who provides support for
operations of the AAA infrastructure.

%XXX Ilya: add EGM reference from the previous proposal
{\bf Reconstruction and calibration}
Group members have contributed to the alignment of the CMS tracker.  Postdoc Frank Meier served as convener of the CMS tracker alignment group in 2013-15.  The tracker must be aligned to a fine degree to reach sub-ten-micron space point measurement precision.  Meier led the effort to achieve an incredible 1-2 micron precision alignment of the individual components of the silicon tracker~\cite{bib:alignment}.  Avdeeva worked with Meier first on aligning the tracker using cosmic rays during the period before proton collisions and later with collision data at the start of Run~2. The group also made major contribution to the work of the CMS EGM Physics Object Group that deals with electron and photon reconstruction, calibration, and triggering on these particles. Kravchenko served as a co-convener of the EGM ID subgroup in 2014-2016. Kravchenko with the graduate student Kamalieddin also produced numerous sets of electron selection packages that were in use in a large number of CMS measurements. This work is reflected on the summary Run-1 paper from the EGM group on reconstruction and calibration of electrons an photons [ref-egm-run-1]. Kravchenko continues to be one of the main EGM experts on ID, while the postdoc Finco is a co-convener of the EGM trigger subgroup since 2018.

{\bf Phase-1 pixel detector ugprade}
An important achievement of the group was the completion of our contribution to the Phase 1 CMS upgrade: the construction of half the modules required for the upgraded forward pixel detector.  This was the culmination of a long effort to develop the capability to do this construction in house, with the support of the department electronics and instrument shops.  The day-to-day efforts were carried out by postdocs Meier and Bartek and graduate students Monroy, Fangmeier and Siado, assisted by a team of undergraudates and shop technicians.  Their task was to take bump-bonded assemblies of silicon pixel sensors and readout chips, and then glue them to high density interconnect boards, wirebond the circuits together, encapsulate the bonds, and test that the assembly was done correctly and make repairs as needed.  This required the commissioning and programming of a computer-controlled assembly gantry and associated tooling and an automated wirebonder and the careful development of a production assembly process.  Over the course of 13 months, the team assembled 555 modules, with enough having sufficient quality for installation in CMS.  This mammoth effort established UNL's capability as a module production site, which will be required for the successful completion of the HL-LHC upgrade.

{\bf Collaboration leadership}
Snow continued his leadership of the 12-member CMS PhD Thesis Award Committee; he founded the program in 2000 and has served as chair of this committee in 2013~\cite{bib:thesisawardwebsite}.  Every year the committee reviews 15-20 theses, a number that has grown over time, and identifies one more award recipients each year.  This committee also oversees a series of Named Lectures highlighting the contributions of outstanding young researchers to the experiment.

Snow served as Chair of the U.S.~CMS Collaboration Board during 2016-18 after being Deputy Chair in 2014-16.  During these periods new leadership of the Operations Program and Fermilab's LHC Physics Center were appointed, requiring the involvement of the Collaboration Board leadership.  Snow was involved in those processes, along with many other search and election committees, planning annual U.S. CMS collaboration meetings, and helping to organize the response to the 2017-18 Department of Energy Office of Science Portfolio Review~\cite{bib:portfolioreview}

\paragraph{DZERO}
The group has now largely closed out its involvement with the DZERO experiment after the Tevatron finished operations in 2011.  The only recent activity has been Bloom's chairmanship of the editorial board that reviewed $t\bar{t}$ forward-backward asymmetry measurements, which were for a period of time of great interest to the community after early measurements gave anomalously large values.  The final DZERO papers on the topic have now been published, along with the legacy Tevatron measurement~\cite{bib:D0afb}.

\subsection{Broader Impacts}

{\bf Needs a CROP etc. section!}

Bloom has continued with a variety of outreach activities.  He served a U.S. LHC-sponsored blogger for the Quantum Diaries website~\cite{bib:bloomblog} until it stopped operating in 2016.  Most prominently, in cooperation with ``The Big Bang Theory'' television program, he wrote a blog post~\cite{bib:TBBTQD} that was used as a major plot point in a February 2015 episode~\cite{bib:TBBTepisode}.  Since the end of Quantum Diaries, he has been writing occasional articles for Fermilab Today~\cite{bib:BloomFNALToday}.  Bloom has given public talks on particle physics and the LHC in particular in a variety of venues, ranging from an invited plenary talk at the Winter 2017 American Association of Physics Teachers meeting~\cite{bib:BloomAAPT} to students in rural Nebraska, such as Cambridge, NE, population 1047, where he met with all 350 students in the K-12 system.

Gonzalez Suarez and Monroy have led the UNL outreach effort at CERN. They have both been official CERN tour guides, conducting tours for the general public at multiple CERN experimental facilities, including the underground caverns of CMS and other experiments. Gonzalez Suarez has also guided virtual tours to the cavern of the CMS experiment for international audiences, moderated several international masterclasses by the International Particle Physics Outreach Group from CERN, and was featured in the LHC Season 2 videos produced by CERN for the start of Run 2~\cite{bib:LHCSeason2}. Gonzalez Suarez was one of the organizers of bi-annual workshops from the series Expanding your Horizons at CERN, an initiative to bring STEM career choices to schoolgirls. 

Kravchenko's work in RICE and ARA experiments, done entirely through the efforts of UNL undergraduate students, provides students with solid research experience, with two to four students involved at any given time over the last five years.  

{\bf Update Auger outreach if Greg's activities fall within the past five years!}

%XXX Greg: if you have any more events of this sort, please add. If not
%        please remove your name from this comment.
{\bf Needs updating by Dan/Greg:}
All faculty members continue to give presentations to public groups in Lincoln and elsewhere in Nebraska about the exciting science of particle physics. These have included over a dozen annual presentations of Claes's popular ``Comic Book Physics 101" series, YouTube videos inspired by these talks and produced by the American Chemical Society's ACS Reactions (featuring Claes as the consulting physics expert), and Claes's Spring 2015 public lecture targeting the general population of Lincoln, ``What the Heck is a Higgs Boson?" part of the UNL Chancellor's Distinguished Lecture Series.  Along with Claes, Snow is a member of UNL's Speakers Bureau, and has given presentations on the Higgs boson and ultra-high energy cosmic rays to civic groups, clubs and schools.
